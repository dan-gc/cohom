\documentclass[spanish]{book}
\usepackage{titlesec}

%Quitar páginas en blanco
\let\cleardoublepage\clearpage
\usepackage{etoolbox}
\makeatletter
\patchcmd{\@endpart}{\vfil\newpage}{\par}{}{}
\makeatother

%\usepackage[spanish]{babel} ¡Esto estaba interfiriendo con las flechitas de los \tikspicture

\renewcommand{\contentsname}{Índice}
\renewcommand{\partname}{Parte}

\titleformat{\chapter}[display]
{\normalfont\huge\bfseries}{}{0pt}{\Huge\thechapter.~}

\titleformat{name=\chapter,numberless}[display]
{\normalfont\huge\bfseries}{}{0pt}{\Huge}
\renewcommand{\chaptermark}[1]{\markboth{{} \thechapter: #1}{}}

\usepackage[left=4cm, right=4cm]{geometry}
\usepackage{palatino}%Font
\usepackage{graphicx}
\usepackage{marvosym}%Smileys
\usepackage{float}
\usepackage{subcaption}
\usepackage{enumitem}
\usepackage{parskip}
\usepackage{amsthm}
\usepackage{amssymb}
\usepackage{amsmath}
\usepackage{stmaryrd}
\usepackage{tikz}
\usepackage{tikz-cd}
\usepackage[bookmarks,bookmarksopen,bookmarksdepth=3]{hyperref}
\hypersetup{
	colorlinks=true,
	urlcolor=blue,
	linkcolor=magenta,
	citecolor=blue,
	filecolor=blue,
	urlbordercolor=white,
	linkbordercolor=white,
	citebordercolor=white,
	filebordercolor=white
}

\theoremstyle{definition}
\renewcommand{\proofname}{Demostración}

\newtheorem*{defn}{Definición}
\newtheorem*{lema}{Lema}
\newtheorem*{obs}{Observación}
\newtheorem*{teo}{Teorema}
\newtheorem*{prop}{Proposición}
\newtheorem*{coro}{Corolario}
\newtheorem*{ejer}{Ejercicio}
\newtheorem*{ejem}{Ejemplo}
\newtheorem*{af}{Afirmación}
\newtheorem*{pregunta}{Pregunta}

\newcommand{\R}{\mathbb{R}}
\newcommand{\Z}{\mathbb{Z}}
\newcommand{\N}{\mathbb{N}}
\newcommand{\C}{\mathbb{C}}
\newcommand{\Q}{\mathbb{Q}}
\DeclareMathOperator{\Id}{Id}
\DeclareMathOperator{\img}{img}
\DeclareMathOperator{\coker}{coker}
\DeclareMathOperator{\ch}{ch}
\DeclareMathOperator{\comp}{comp}
\DeclareMathOperator{\RCH}{R\text{-}ch\text{-}comp}
\DeclareMathOperator{\RMod}{R\text{-}mod}
\DeclareMathOperator{\ZMod}{\mathbb{Z}\text{-}mod}
\DeclareMathOperator{\Ab}{Ab}
\DeclareMathOperator{\Hom}{Hom}
\DeclareMathOperator{\Cohom}{Cohom}
\DeclareMathOperator{\Ext}{Ext}
\DeclareMathOperator{\Tor}{Tor}



\title{La conjetura de Poincaré en dimensiones altas}
\author{Notas\\ \\ \href{https://github.com/danimalabares/cohom}{github.com/danimalabares/cohom}}

\begin{document}
	\maketitle
	\phantomsection
	\addcontentsline{toc}{part}{\contentsname}
	\tableofcontents
	
\chapter{Ágebra homológica}
\section{Repaso}
Sea $R$ un anillo asociativo con 1. Podemos ahora tomar la categoría de $R$-módulos, $R$-$\mod$, cuyos objetos son $R$-módulos y los morfismos son homomorfismos $R$-lineales. También podemos construir $\RCH$, cuyos objetos son complejos de cadenas,
\[\begin{tikzcd}
	\cdots\arrow[r,"\partial_{n+1}"]&C_n\arrow[r,"\partial_{n}"]&C_{n-1}\arrow[r,"\partial_{n-1}"]&\cdots
\end{tikzcd}\]
tales que $\partial_{n-1}\circ\partial_n=0$, es decir, $\img\partial_n\subseteq\ker\partial_{n-1}$.
y sus morfismos son morfismos complejos de cadenas, \begin{tikzcd}
	C_\bullet\arrow[r,"f"]&D_\bullet\end{tikzcd},
que son muchos morfismos tales que el siguiente diagrama conmuta en todos los cuadraditos:
\[\begin{tikzcd}
	\cdots \arrow{r} & C_{n+1} \arrow{r}{\partial_{n+1}} \arrow{d}{f_{n+1}} & C_n \arrow{r}{\partial_n} \arrow{d}{f_n} & C_{n-1} \arrow{r} \arrow{d}{f_{n-1}} & \cdots \\
	\cdots \arrow{r} & D_{n+1} \arrow{r}{\delta_n} & D_n \arrow{r}{\delta_n} & D_{n-1} \arrow{r} & \cdots
\end{tikzcd}\]
Y definimos
\[H_n(C_\bullet)=\frac{\ker\partial_n}{\img\partial_{n+1}}\]
\begin{defn}\leavevmode
	\begin{itemize}
		\item Decimos que $C_\bullet$ es \textbf{acíclico} si $H_n(C_\bullet)=0$ para toda $n$.
		\item La sucesión
		\begin{tikzcd}[column sep=small]
			C_1\arrow[r,"\varphi"]&C_2\arrow[r,"\psi"]&C_3
		\end{tikzcd}
		es \textbf{exacta} si $\img\varphi=\ker\psi$.
		\item La sucesión 
		\begin{tikzcd}[column sep=small]
			0\arrow[r]&C_1\arrow[r]&C_2\arrow[r]&C_3\arrow[r]&0
		\end{tikzcd}
		es una \textbf{sucesión exacta corta}.
		\item Y si se extiende infinitamente, es una \textbf{sucesión exacta larga}.
	\end{itemize}
\end{defn}
\begin{prop} En una sucesión exacta corta, $\varphi$ es inyectiva, $\psi$ es suprayectiva y $C_3\approx C_2/\ker\psi$. Abusando de notación, podemos pensar que $C_3\approx C_2/C_1$, pero hay que tener cuidado aquí porque el encaje de $C_1$ en $C_2$ puede no ser único.
\end{prop}
Tomemos $n$ fijo. Entonces
\[\begin{tikzcd}
	&\Ab\\
	\RCH \arrow[r,"H_n"]\arrow[ru]&\RMod \arrow[u]\\[-0.6cm]
	C_\bullet\arrow[r,maps to]&H_n(C_\bullet)
\end{tikzcd}\]
Y como los morfismos de cadenas mandan ciclos en ciclos y fronteras en fronteras, podemos definir los morfismos inducidos, que satisfacen que $(fg)_*=f_*g_*$ y $id_{C_{\bullet*}}=id_{H_n(C_\bullet)}$. Como la composición de morfismos se abre en el mismo orden en el que estaba, se llama \textbf{funtor covariante}.
\begin{defn}
	Dos homomorfismos 
	\begin{align*}
		f,g:(C_\bullet,\partial)\to(C'_\bullet,\partial')
	\end{align*}
	son \textbf{homotópicos} si existen homomorfismos $h_n:C_n\to C'_{n+1}$ para toda $p\in\Z$ tales que \[f_n-g_n=\partial'_{n+1}h_n+h_{n-1}\partial_n\] Estas flechas se pueden visualizar aquí:
	\[\begin{tikzcd}[column sep=large, row sep=large]
		\cdots \arrow{r} & C_{n+1} \arrow{r}{\partial_{n+1}} \arrow{d}[left]{f_{n+1}-g_{n+1}} & C_n \arrow{r}[blue]{\partial_n} \arrow{d}[right,red]{f_n-g_n} \arrow{ld}[left,blue]{h_n} & C_{n-1} \arrow{r} \arrow{d}[right]{f_{n-1}-g_{n-1}} \arrow{ld}[right,blue]{h_{n-1}} & \cdots \\
		\cdots \arrow{r} & C'_{n+1} \arrow{r}[below,blue]{\partial'_{n+1}} & C'_n \arrow{r}[below]{\partial_n'} & C'_{n-1} \arrow{r} & \cdots
	\end{tikzcd}
	\]
	Así que la suma de las flechas azules es igual a la flecha roja. (No estamos diciendo que el diagrama sea conmutativo).
	
	Esto es tanto como decir que $H_n(f)=H_n(g)$ para toda $n$. Es decir, funciones homotópicas inducen los mismos homomorfismos entre complejos de cadenas.
\end{defn}
	\begin{teo}[fundamental del álgebra homológica]
	Si 
	\[\begin{tikzcd}[column sep=small]
		0\arrow{r}&A_\bullet\arrow{r}{\phi}&B_\bullet\arrow{r}{\psi}&C_\bullet\arrow{r}&0
	\end{tikzcd}\]
	es una sucesión exacta corta de complejos de cadena, entonces existen homomorfismos \[\delta_{*p}:H_p(C_\bullet)\to H_{p-1}(A_\bullet)\]
	tales que la sucesión
	\[\begin{tikzcd}[column sep=small]
		&\cdots\arrow{r}&H_p(A_\bullet)\arrow{r}{\bar\phi_p}&H_p(B_\bullet)\arrow{r}{\bar\psi_p}&H_p(C_\bullet)\arrow{r}{\delta_{*p}}&H_{p-1}(A_\bullet)\arrow{r}{\bar\phi_{p-1}}&H_{p-1}(B_\bullet)\arrow{r}&\cdots
	\end{tikzcd}\]
	es exacta.
\end{teo}
En el siguiente diagrama conmutativo se ve claramente qué está pasando:
\[
\begin{tikzcd}
	& & 0 \arrow{d} & 0 \arrow{d} & 0 \arrow{d} & \\
	& \cdots \arrow{r} & A_{p+1} \arrow{r}{\partial_{p+1}} \arrow{d}{i_{p+1}} & A_p \arrow{r}{\partial_p} \arrow{d}{i_p} & A_{p-1} \arrow{r} \arrow[d,magenta,"i_{p-1}"] & \cdots \\
	& \cdots \arrow{r} & B_{p+1} \arrow{r}{\partial_{p+1}} \arrow{d}{j_{p+1}} & B_p \arrow[r,magenta,"\partial_p"] \arrow[d,magenta,"j_p"] & B_{p-1} \arrow{r} \arrow{d}{j_{p-1}} & \cdots \\
	& \cdots \arrow{r} & C_{p+1} \arrow{r}{\partial_{p+1}} \arrow{d} & C_p \arrow{r}{\partial_p} \arrow{d} & C_{p-1} \arrow{r} \arrow{d} & \cdots \\
	& & 0 & 0 & 0 & \\
\end{tikzcd}
\]
\begin{proof}
	Explicamos un poco cómo definir el homomorfismo de conexión haciendo cacería de diagrama. Comenzamos con un ciclo $c\in C_p(A)$. Como $j_p$ es suprayectiva, existe un $a\in B_p$ tal que $j_p(a)=c$. Luego, $\partial_p(a)\in\ker j_{p-1}$, ya que, como el diagrama conmuta, $\partial_pj_p=j_{p-1}\partial_p$ y $c$ es un ciclo. Como la sucesión es exacta, $\ker j_{p-1}=\img i_{p-1}$, así que existe $a\in A_{p-1}$ tal que $i_{p-1}(a)=\partial_p(b)$. Este $a$ es un ciclo, ya que el diagrama conmuta, $i_{p-2}(a)=\partial(\partial(b))=0$, y la $i_{p-2}$ es inyectiva por exactitud, es decir, el único elemento al que va a dar el cero es el cero. Así que definimos $\delta_{*p}[c]=[a]$.
	
	Y una vez definido este homomorfismo, el resto de la prueba sale sin trucos.
\end{proof}
\begin{teo}[Naturalidad del homomorfismo de conexión]
	Para dos sucesiones exactas cortas y morfismos $f$, $g$ y $h$,
	\[\begin{tikzcd}[column sep=small]
		&0\arrow{r}&A_\bullet\arrow{r}{i}\arrow{d}{f}&B_\bullet\arrow{r}{j}\arrow{d}{g}&C_\bullet\arrow{r}\arrow{d}{h}&0\\
		&0\arrow{r}&A'_\bullet\arrow{r}&B'_\bullet\arrow{r}&C'_\bullet\arrow{r}&0
	\end{tikzcd}\]
	donde las filas son exactas.\par
	Entonces, el siguiente diagrama conmuta
	\[\begin{tikzcd}[column sep=small, row sep=large]
		& \cdots \arrow{r} & H_p(A) \arrow{r} \arrow{d}{\bar{f}} & H_p(B) \arrow{r} \arrow{d}{\bar{g}} & H_p(C) \arrow{r}{\delta_*} \arrow{d}{\bar{h}} & H_{p-1}(A) \arrow{r} \arrow{d}{\bar{f}} & H_{p-1}(B) \arrow{r} \arrow{d}{\bar{g}} & H_{p-1}(C) \arrow{r} \arrow{d}{\bar{h}} & \cdots \\
		& \cdots \arrow{r} & H_p(A') \arrow{r} & H_p(B') \arrow{r} & H_p(C') \arrow{r} & H_{p-1}(A') \arrow{r} & H_{p-1}(B') \arrow{r} & H_{p-1}(C') \arrow{r} & \cdots
	\end{tikzcd}\]
\end{teo}
\begin{proof}
	Salvo en los cuadrados donde está $\bar{h}$ a la izquierda y $\bar{f}$ a la derecha, la conmutatividad se sigue por funtorialidad.
\end{proof}
\begin{lema}[de los cinco]
	Consideremos el diagrama conmutativo con filas exactas
	\[\begin{tikzcd}
		M_5\arrow{r}{f_5}\arrow{d}{h_5}&M_4\arrow{r}{f_4}\arrow{d}{h_4}&M_3\arrow{r}{f_3}\arrow{d}{h_3}&M_2\arrow{r}{f_2}\arrow{d}{h_2}&M_1\arrow{d}{h_1}\\
		N_5\arrow[r,swap,"g_5"]&N_4\arrow[r,swap,"g_4"]&N_3\arrow[r,swap,"g_3"]&N_2\arrow[r,swap,"g_2"]&N_1
	\end{tikzcd}\]
	Si $h_5,h_4,h_2$ y $h_1$ son isomorfismos, entonces $h_3$ también.
\end{lema}

\section{Más álgebra homológica}
Tomemos $N,M$ $R$-módulos, y el conjunto de homomorfismos $R$-lineales de $M$ en $N$, que es un grupo abeliano (cuya identidad es el morifsmo que manda todo a $0$, y $f+g(m)=f(m)+g(m)$ que también es un morfismo, $(-f)(m)=-f(m)$). También tiene estructura de $R$-módulo con la operación $(rf)(m)=rf(m)=f(rm)$.

Ahora construyamos un funtor:
\begin{align*}
	\Hom(-,N):\RMod&\to\Ab\\
	M&\mapsto\Hom(M,N)\\
	M\xrightarrow{\varphi} M'&\mapsto \qquad?
\end{align*}
La flecha inducida será
\begin{align*}
	\Hom(M,N)&\xleftarrow{\varphi^*}\Hom(M',N)\\
	\varphi^*(f)=f\varphi&\mapsfrom f
\end{align*}
De acuerdo a
\[\begin{tikzcd}
	M\arrow[d,"\varphi"]\arrow[rd,"f\varphi"]\\
	M'\arrow[r,swap,"f"]&N
\end{tikzcd}\]
Así que $\Hom(-,N)$ es un funtor \textbf{contravariante}. De hecho, es un \textbf{funtor aditivo exacto izquierdo}:
\begin{itemize}
	\item \textbf{Aditivo}. Manda sumas directas en sumas directas, es decir,
	\[\Hom(M_1\oplus M_2,N)\approx \Hom(M_1,N)\oplus\Hom(M_2,N)\]
	Que tiene que ver con la propiedad universal de la suma directa:
	\[\begin{tikzcd}
		M_1\arrow[r,hook]\arrow[rd,"f"]&M_1\oplus M_2\arrow[d,dashed,"\exists! f\oplus g"]&M_2\arrow[l,hook']\arrow[dl,"g"]\\
		&N
	\end{tikzcd}\]
	Donde $(f\oplus g)(m_1,m_2)=f(m_1)+g(m_2)$. Así que si tenemos $(f,g)$ en el módulo de la derecha, lo mandamos a $f\oplus g$.
	\item \textbf{Exacto} Supongamos que tenemos la sucesión exacta corta
	\[\begin{tikzcd}
		0\arrow[r]&A\arrow[r,"\varphi"]&B\arrow[r,"\psi"]&C\arrow[r]&0
	\end{tikzcd}\]
	a la que le aplicamos el funtor para obtener la sucesión exacta
	\[\begin{tikzcd}
		0\arrow[r]&\Hom(C,N)\arrow[r,"\psi^*"]&\Hom(B,N)\arrow[r,"\varphi^*"]&\Hom(A,N)
	\end{tikzcd}\]
	En general, $\varphi^*$ no es suprayectiva.
\end{itemize}

\begin{ejer}\leavevmode
	\begin{itemize}
		\item Checar lo anterior.
		\begin{proof}[Solución]
			Basta ver que la flecha $0\to A$ no necesiamente va a dar a una flecha de la forma $\Hom(A,N)\to0$.
		\end{proof}
		\item ¿Qué pasa con el cokernel?
	\end{itemize}
\end{ejer}
\begin{obs}
	También podemos definir el funtor análogo dejando libre la entrada de la derecha, y obtenemos un funtor covariante (que no usaremos tanto y también es aditivo exacto \textit{izquierdo}).
\end{obs}
\begin{obs}
	Denotaremos $\Hom_R(M,N):=M^*$, y, por si acaso $\Hom(N,M):=M_*$.
\end{obs}

\section{Funtores derivados}
Es un juego, y usaremos $R$-módulos libres, que tienen la ventaja de tener una base. Un \textbf{$R$-módulo libre} es uno de la forma $\bigoplus_{i\in I}R_i$ donde $R_i=R$. Los elementos canónicos son $e_j:=(\delta_{ij})_{i\in I}$, y $\beta:=\{ej\}_{j\in J}$ es una \textbf{base} en cuanto a que cumple la siguiente propiedad universal: para cualquier $R$-módulo $M$ y para toda función $f:\beta\to M$ existe un único $\bar{f}:L=\bigoplus_{i\in I}R_i\to M$ tal que el siguiente diagrama conmuta:
\[\begin{tikzcd}
	\beta\arrow[r,"f"]\arrow[d,hook]&M\\
	L\arrow[ur,swap,dashed,"\bar{f}"]
\end{tikzcd}\]
Luego, diremos que $P$ es \textbf{proyectivo} si existe $Q$ tal que $P\oplus Q$ es libre.
\begin{ejem}
	$\Z/6\approx \Z/2\oplus\Z/3$
\end{ejem}
\begin{prop}
	Todo $R$-módulo es cociente de un $R$-módulo libre. 
\end{prop}
\begin{proof}
	\[\begin{tikzcd}
		L=\bigoplus_{i\in M}R_i\arrow[r,dashed,"\bar{f}"]&M\\
		M\arrow[u,hook]\arrow[ur]
	\end{tikzcd}\]
	Como $\bar{f}$ es suprayectiva, por primer teorema de isomorfismo, terminamos.
\end{proof}
\begin{defn}
	Sea $M$ un $R$-módulo. Una \textbf{resolución libre (proyectiva)} de $M$ es una sucesión exacta de la forma
	\[\begin{tikzcd}
		\cdots\arrow[r]&F_1\arrow[r,"f_1"]&F_0\arrow[r,"f_0"]&M\arrow[r]&0
	\end{tikzcd}\]
	tal que $F_j$ es libre para toda $j$.
\end{defn}
\begin{teo}
	Todo $R$-módulo tiene una resolución libre (proyectiva).
\end{teo}
\begin{proof}
	$f_0$ sale por la proposición anterior. Tomamos el módulo $\ker f_0$, lo incluimos en $F_0$ escogemos $F_1$ que cubre $\ker f_0$ por la proposición anterior.
	\[\begin{tikzcd}
		\cdots\arrow[r]&F_1\arrow[rd,"f_1"]\arrow[r,"f_1"]&F_0\arrow[r,"f_0"]&M\arrow[r]&0\\
		&&\ker f_0\arrow[u,hook]&&
	\end{tikzcd}\]
\end{proof}
\begin{teo}
	Sea $\alpha:M\to M'$ un homomorfismo.
	\[\begin{tikzcd}
		\cdots\arrow[r]&F_2\arrow[r,"f_2"]\arrow[d,dashed,"\alpha_2"]&F_1\arrow[r,"f_1"]\arrow[d,dashed,"\alpha_1"]&F_0\arrow[d,dashed,"\alpha_0"]\arrow[r,"f_0"]&M\arrow[d,"\alpha"]\arrow[r]&0\\
		\cdots\arrow[r]&F'_2\arrow[r,"f_2"]&F'_1\arrow[r,"f_1"]&F'_0\arrow[r,"f_0"]&M'\arrow[r]&0
	\end{tikzcd}\]
	entonces existen los $\alpha_i$ que hacen conmutar el diagrama.
	
	Más aún, si existen $\beta_i:F_i\to F'_i$ que cumplen lo mismo entonces los homomorfismos determinados por los $\alpha_i$ y $\beta_i$ son homotópicos.
\end{teo}
\begin{proof}
	\[\begin{tikzcd}
		&\beta_0\arrow[d,hook,]&&\\	
		\cdots\arrow[r]&F_0\arrow[d,dashed,"\alpha_0"]\arrow[r,"f_0"]&M\arrow[d,"\alpha"]\arrow[r]&0\\
		\cdots\arrow[r]&F'_0\arrow[r,"f'_0"]&M\arrow[r]&0
	\end{tikzcd}\]
	Tomamos un elemento $e\in \beta_0$ en la base de $F_0$. Lo mandamos mediante $f_0$ a $M$, luego con $\alpha$. Pero como $f'_0$ es supra, podemos escoger un elemento $e'\in F'_0$ que le pega.
	
	Ahora
	\[\begin{tikzcd}
		&&&\img f_1\arrow[d,hook,]&&\\	
		\cdots\arrow[r]&F_2\arrow[r,"f_2"]\arrow[d,dashed,"\alpha_2"]&F_1\arrow[ur]\arrow[r,"f_1"]\arrow[d,dashed,"\alpha_1"]&F_0\arrow[d,dashed,"\alpha_0"]\arrow[r,"f_0"]&M\arrow[d,"\alpha"]\arrow[r]&0\\
		\cdots\arrow[r]&F'_2\arrow[r,"f'_2"]&F'_1\arrow[dr]\arrow[r,"f'_1"]&F'_0\arrow[r,"f'_0"]&M\arrow[r]&0\\
		&&&\img f'_1\arrow[u,hook,]&&
	\end{tikzcd}\]
	Hay una flecha desde $\img f_1$ hasta $\img f'_1$ que cierra el diagrama.
	
	Faltó lo de homotopía (usando las diagonales como el diagrama coloreado).
\end{proof}
\begin{defn}
	Sea $M$ un $R$-módulo. Tomamos una resolución libre
	\[\begin{tikzcd}
		\cdots\arrow[r]&F_1\arrow[r,"f_1"]&F_0\arrow[r,"f_0"]&M\arrow[r]&0
	\end{tikzcd}\]
	Quitamos $M$:
	\[\begin{tikzcd}
		\cdots\arrow[r]&F_1\arrow[r,"f_1"]&F_0\arrow[r]&0
	\end{tikzcd}\]
	Aplicamos $\Hom_R(-,N)$:
	\[\begin{tikzcd}
		0\arrow[r]&F_0^*\arrow[r,"f_1"]&F_1^*\arrow[r]&F_2^*\arrow[r,"f_3^*"]&\cdots
	\end{tikzcd}\]
	Definimos
	\[\Ext^n_R(M,N):=H_n(0\to F^*)\]
\end{defn}
\begin{teo}
	$\Ext^n_R(M,N)$ no depende de la resolución.
\end{teo}
\begin{proof}
	Usamos el teorema anterior (dos veces), tomando dos resoluciones de $M$ usando la identidad como $\alpha$:
	\[\begin{tikzcd}
		\cdots\arrow[r]&F_2\arrow[r,"f_2"]\arrow[d,"\alpha_2"]&F_1\arrow[r,"f_1"]\arrow[d,"\alpha_1"]&F_0\arrow[d,"\alpha_0"]\arrow[r,"f_0"]&M\arrow[d,"Id"]\arrow[r]&0\\
		\cdots\arrow[r]&F'_2\arrow[r,"f_2"]\arrow[d,"\beta_2"]&F'_1\arrow[r,"f_1"]\arrow[d,"\beta_1"]&F'_0\arrow[d,"\beta_0"]\arrow[r,"f_0"]&M\arrow[d,"Id"]\arrow[r]&0\\
		\cdots\arrow[r]&F_2\arrow[r,"f_2"]&F_1\arrow[r,"f_1"]&F_0\arrow[r,"f_0"]&M\arrow[r]&0
	\end{tikzcd}\]
	Y aquí resulta que $\{\beta_i\alpha_i\}\simeq\{Id\}$. Y dualizamos:
	\[\begin{tikzcd}
		\cdots&F_1^*\arrow[l]&F_0^*\arrow[l]&M^*\arrow[l]&0\arrow[l]\\
		\cdots&F_1^{'*}\arrow[u,"\alpha_1^*"]\arrow[l]&F^{'*}_0\arrow[u,"\alpha_0^*"]\arrow[l]&M^*\arrow[l]&0\arrow[l]\\
		\cdots&F_1^{'*}\arrow[u,"\beta_1^*"]\arrow[l]&F^{'*}_0\arrow[u,"\beta_0^*"]\arrow[l]&M^*\arrow[l]&0\arrow[l]
	\end{tikzcd}\]
	Y como el funtor es aditivo, la homotopía pasa al dual, es decir, $\{\beta_i^*\alpha_i^*\}\simeq\{Id\}$. Luego pasamos a los grupos de homología:
	\[\begin{tikzcd}
		H_1(F^*)\\
		H_1(F'^*)\arrow[u,"\alpha_1^\#"]\\
		H_1(F^*)\arrow[u,"\beta_1^\#"]
	\end{tikzcd}\]
	Cambiando los roles, obtenemos que estas dos funciones $\alpha_1^\#$ y $\beta_1^\#$ son inversas una de la otra.
\end{proof}\vspace{-.3cm}
\begin{prop}
	$\Ext^0_R(M,N)\approx\Hom_R(M,N)$.
\end{prop}
\begin{proof} Tenemos:
	\[\begin{tikzcd}
		0\arrow[r]&\img f_1\arrow[r,"f_1"]&F_0\arrow[r,"f_0"]&M\arrow[r]&0
	\end{tikzcd}\]
	\[\begin{tikzcd}
		0\arrow[r]&M^*\arrow[r,"f_0^*"]&F^*_0\arrow[r,"f_1^*"]&\img f_1^*
	\end{tikzcd}\]
	Luego $\ker f_1^*\approx \img f_0^*\approx M^*$. Luego, por definición $\Ext^0_R(M,N)=\ker f_1^*\approx M^*=\Hom(M,N)$
\end{proof}

\begin{lema}[De la herradura] \textit{Sucesiones exactas cortas de módulos inducen sucesiones exactas cortas de resoluciones.} Tomemos una sucesión exacta corta y dos resulciones libres de los extremos. Entonces existe lo rojo:
	\[\begin{tikzcd}
		&0&0&0\\
		1\arrow[r]&M'\arrow[u]\arrow[r]&M\arrow[red,u]\arrow[r]&M''\arrow[u]\arrow[r]&0\\
		0\arrow[red,r]&F_0'\arrow[r,red]\arrow[u]&\textcolor{red}{F_0}\arrow[r,red]\arrow[u,red]&F_0^{''}\arrow[u]\arrow[r,red]&0\\
		0\arrow[red,r]&F_1'\arrow[r,red]\arrow[u]&\textcolor{red}{F_1}\arrow[r,red]\arrow[u,red]&F_1^{''}\arrow[u]\arrow[r,red]&0\\
		&\vdots\arrow[u]&\vdots\arrow[u,red]&\vdots\arrow[u]&
	\end{tikzcd}\]
\end{lema}
\begin{proof} Pa' pronto, la resolución de en medio es la suma de las resoluciones:
	\[\begin{tikzcd}
		&0&&0\\
		1\arrow[r]&M'\arrow[r]\arrow[u]&M\arrow[r]&M''\arrow[r]\arrow[u]&0\\
		0\arrow[red,r]&F_0'\arrow[r,red]\arrow[u]&F_0'\oplus F_0^{''}\arrow[r,red]\arrow[u,red]&F_0^{''}\arrow[u]\arrow[r,red]&0
	\end{tikzcd}\]
	Y hay que hacer todo lo de rutina.
\end{proof}
Ahora apliquemos $\Hom(-,N)$:
\[\begin{tikzcd}
	&0\arrow[d]&0\arrow[d]&0\arrow[d]&\\
	0&P_0^{'*}\arrow[l]\arrow[d]&P_0^{*}\arrow[l]\arrow[d]&P_0^{''*}\arrow[l]\arrow[d]&0\arrow[l]\\	0&P_1^{'*}\arrow[l]\arrow[d]&P_1^{*}\arrow[l]\arrow[d]&P_1^{''*}\arrow[l]\arrow[d]&0\arrow[l]\\
	&\cdots&\cdots&\cdots&
\end{tikzcd}\]
O sea que tenemos la sucesión exacta corta
\[\begin{tikzcd}
	0&P^{'*}_\bullet\arrow[l]&P^*_\bullet\arrow[l]&P_\bullet^{''*}\arrow[l]&0\arrow[l]
\end{tikzcd}\]
A la que aplicamos el teorema fundamental del álgebra homológica para obtener
\[\begin{tikzcd}
	0\arrow[r]&\Ext_R^0(M'',N)\arrow[r]&\Ext_R^0(M,N)\arrow[r]&\Ext_R^0(M',N)\arrow[r]&\Ext_R^1(M'',N)\arrow[r]&\cdots
\end{tikzcd}\]
Y notemos que los primeros tres módulos son:
\[\begin{tikzcd}
	0\arrow[r]&\Hom_R(M'',N)\arrow[r]&\Hom_R(M,N)\arrow[r]&\Hom_R(M',N)\arrow[r]&\Ext_R^1(M'',N)\arrow[r]&\cdots
\end{tikzcd}\]
Y ahora supongamos que tenemos
\[\begin{tikzcd}
	0\arrow[r]&M'\arrow[r]\arrow[d]&M\arrow[r]\arrow[d]&M''\arrow[r]\arrow[d]&0\\
		0\arrow[r]&A'\arrow[r]&A\arrow[r]&A''\arrow[r]&0
\end{tikzcd}\]
Que inducen
\[\begin{tikzcd}
	0&P^{'*}_\bullet\arrow[l]&P^*_\bullet\arrow[l]&P_\bullet^{''*}\arrow[l]&0\arrow[l]\\
		0&Q^{'*}_\bullet\arrow[l]\arrow[u]&Q^{*}_\bullet\arrow[l]\arrow[u]&Q_\bullet^{''*}\arrow[l]\arrow[u]&0\arrow[l]
\end{tikzcd}\]
Y por fin obtenemos
\[\begin{tikzcd}
	0\arrow[r]&\Ext_R^0(M'',N)\arrow[r]&\Ext_R^0(M,N)\arrow[r]&\Ext_R^0(M',N)\arrow[r]&\Ext_R^1(M'',N)\arrow[r]&\cdots\\
		0\arrow[r]&\Ext_R^0(A'',N)\arrow[r]\arrow[u]&\Ext_R^0(A,N)\arrow[r]\arrow[u]&\Ext_R^0(A',N)\arrow[r]\arrow[r]&\Ext_R^1(M'',N)\arrow[r]\arrow[u]&\cdots
\end{tikzcd}\]
donde todo conmuta. Y ese es más o menos el juego de los funtores derivados.

Es momento de hacer un decreto:
\begin{quotation}
	\textbf{A partir de ahora el anillo será $\Z$.}
\end{quotation}
Tomemos entonces $A,B\in\ZMod$ y el funtor $\Hom_\Z(-,B)$. ¿Cómo será una resolución libre proyectiva para $A$?
\[\begin{tikzcd}
	0\arrow[r]&0\arrow[r]&\ker f_0=P_1\arrow[r]&\bigoplus_{I_0}\Z=P_0\arrow[r,"f_0"]&A\arrow[r]&0
\end{tikzcd}\]
Que inducen
\[\begin{tikzcd}
	0&0\arrow[l]&P_1^*\arrow[l]&P_0^*\arrow[l]&0\arrow[l]
\end{tikzcd}\]
Es decir,
\begin{quotation}
	$\Ext^n_\Z(A,B)=0$ para cualesquiera $\Z$-módulos $A,B$ y $n\geq2$.
\end{quotation}
\begin{prop}\leavevmode
	\begin{itemize}
		\item $\Ext^n_R(-,N)$ es un funtor aditivo para toda $n$, es decir,
		\[\Ext_R^n(M'\oplus M'',N)\approx\Ext_R^n(M',N)\oplus\Ext^n_R(M'',N)\]
		\begin{proof}
			Consideremos
			\[\begin{tikzcd}[row sep=small]
				P'_\bullet\arrow[r]&M'\arrow[r]&0\\
				P''_\bullet\arrow[r]&M''\arrow[r]&0
			\end{tikzcd}\]
			Y no es difícil ver que también tenemos
			\[\begin{tikzcd}
				P'_\bullet\oplus P''_\bullet\arrow[r]&M'\oplus M''\arrow[r]&0
			\end{tikzcd}\]
			Y la homología abre sumas: $(P_\bullet''\oplus P_\bullet'')^*\approx P_\bullet''\oplus P_\bullet'^*$.
		\end{proof}
		\item $\Ext^n_\Z(A,B)=0$ si $A$ es libre.
		\begin{proof}
			Simplemente tomamos $0\to A\to A\to 0$.
		\end{proof}
		\item $\Ext^1_\Z(\Z/n\Z,B)\approx B/nB$.
		\begin{proof}
			\[\begin{tikzcd}[row sep=small]
				0&\Hom(\Z,B)\arrow[l]\arrow[d,"\approx"]&\Hom(\Z,B)\arrow[l,swap,"n^*"]\arrow[d,"\approx"]&0\arrow[l]\\
				0&B\arrow[l]&B\arrow[l,"n"]&0\arrow[l]
			\end{tikzcd}\]
			Así que $B/nB=\Ext_\Z^1(\Z/n,B)$.
		\end{proof}
	\end{itemize}
\end{prop}
¿Qué obtenemos de esta proposición? Si $A$ es finitamente generado, $A\approx\Z^r\oplus\Z/m_1\oplus\ldots\oplus\Z/m_t$, entonces $\Ext^1_\Z(A,B)\approx B/m_1B\oplus\ldots\oplus B/m_tB$.

\section{Grupos de cohomología}
Tomemos un grupo abeliano $G$ y un complejo de cadenas de grupos abelianos libres
\[\begin{tikzcd}
	\cdots\arrow[r]&C_n\arrow[r,"\partial_n"]&C_{n-1}\arrow[r,"\partial_{n-1}"]&C_{n-2}\arrow[r]&\cdots
\end{tikzcd}\]
Y apliquemos el $\Hom_\Z(-,G)$ para obtener
\[\begin{tikzcd}
	\cdots\arrow[r]&C_n^*\arrow[r,"\partial_{n+1}^*"]&C_{n+1}^*\arrow[r,"\partial^*_{n+2}"]&C_{n+2}^*\arrow[r]&\cdots
\end{tikzcd}\]
que tiene su homología,
\[H^n(C_\bullet,G)=\ker\partial_n^*/\img\partial^*_{n-1}\]
que llamaremos el \textbf{$n$-ésimo grupo de cohomología de $C_\bullet$ con coeficientes en $G$}.

Consideremos
\[\begin{tikzcd}
	\Hom(C_{n-1},G)\arrow[r,"\partial_{n-1}^*"]&\Hom(C_{n-1},G)\arrow[r,"\partial_n^*"]&\Hom(C_n,G)
\end{tikzcd}\]
Y también
\[\begin{tikzcd}
	C_{n-2}\arrow[r,"f"]&G\\
	C_{n-1}\arrow[u,"\partial_{n-1}"]\arrow[ur,"f\partial_{n-1}",swap]
\end{tikzcd}\]
De manera que los elementos en la homología funciones que se anulan en las fronteras, ya que $[g]\in H^{n-1}(C;G)$ para alguna $g:C_{n-1}\to G$ tal que
\begin{align*}
	f\partial_{n-1}:C_{n-1}&\to G\\
	g&\mapsto g\partial_n=0
\end{align*}
\section{Teorema de coeficientes universales}
	Los funtores homología y dualizar no conmutan, es decir $H^n(C_\bullet;G)$ y $\Hom(H_n(C_\bullet),G)$ no son iguales.
\begin{ejem}
	Analizar el caso de 
	\[\begin{tikzcd}
		C_\bullet&0\arrow[r]&\Z\arrow[r,"2"]&\Z\arrow[r]&0
	\end{tikzcd}\]
	comparando $H^n(C_\bullet;\Z)$ con $\Hom(H_n(C_\bullet);\Z)$.
	Para calcular la cohomología lo primero que hago es dualizar:
	\[\begin{tikzcd}
		C^*_\bullet&0&\Z\arrow[l,swap,"2"]&\Z\arrow[l]&0\arrow[l]
	\end{tikzcd}\]
	De manera que $H^0(C_\bullet;\Z)=0$ y $H^1(C_\bullet;\Z)=\Z/2\Z$, pero $\Hom(H_n(C_\bullet),\Z)=0$ para toda $n$.
\end{ejem}
Aún así, podemos construir una función
\begin{align*}
	h:H^n(C_\bullet;G)&\to\Hom(H_n(C_\bullet),G)\\
	[g]&\mapsto Z_n/B_n=H_n(C_\bullet)\to G
\end{align*}
donde $g:C_n\to G$ con $g|_{B_n}=0 $. Así que simplemente enviamos a $[g]$ a la restricción $g|_{Z_n}:Z_n/B_n\to G$.
\begin{prop}
	$h$ es suprayectiva. Más aún, exste una función 
	\[\varphi:\Hom(H_n(C_\bullet),G)\to H^n(C_\bullet;G)\]
	tal que $h\varphi=id$.
\end{prop}
\begin{proof}
	Sea $\bar{g}:H_n(C_\bullet)\to G$. Como $H_n(C_\bullet)=Z_n/B_n$, lo que queremos hacer es extender $\bar{g}$ a una función en todo $C_\bullet$. Observemos que tenemos esta sucesión exacta corta:
	\[\begin{tikzcd}
		0\arrow[r]&Z_n\arrow[r]&C_n\arrow[r]&B_{n-1}\arrow[r]&0
	\end{tikzcd}\]
	Como $B_{n-1}\subset C_{n-1}$ y $C_n$ es libre porque \textbf{estamos suponiendo que $C_\bullet$ es un complejo de cadenas de grupos abelianos libres}. Luego, esta sucesión exacta corta se escinde así que $C_n=Z_n\oplus B_n$, y la proyección a $Z_n$ es un mapeo $C_n\to Z_n$. En fin,
	\[\begin{tikzcd}
		Z_n\arrow[r]&H_n(C)\arrow[r,"\bar{g}"]&G\\
		C_n\arrow[u]\arrow[rru,swap,dashed,"g"]
	\end{tikzcd}\]
	Y de hecho$g$ sí representa un elemento en la cohomología, ya que $g|_{B_n}=0$ porque la flecha de $Z_n\to H_n(C)=Z_n/B_n$ es el paso al cociente, así que se pierden los elementos de $B_n$. Además, $\varphi$ es un homomorfismo. Ya además $\varphi h=id$ por la forma en la que fue contruida: no hemos hecho más que extender y luego restringir una función.
\end{proof}
\begin{coro}
	$H^n(C_\bullet,G)\approx\Hom(H_n(C_\bullet),G)\oplus\ker h$.
\end{coro}
\begin{prop}
	$\ker h\approx \Ext^1_\Z(H_{n-1},G)$.
\end{prop}
\begin{proof}
	Esta prueba tiene un truco. Comenzamos por definir el complejo de cadenas de los ciclos,
	\[\begin{tikzcd}
		0\arrow[d]&&&\\
		Z_\bullet\arrow[d,hook]&\cdots\arrow[r]&Z_n\arrow[d,hook]\arrow[r,"0"]&Z_{n-1}\arrow[d,hook]\arrow[r,"0"]&\cdots\\
		C_\bullet\arrow[d]&\cdots\arrow[r]&C_n\arrow[d,hook]\arrow[r,"\partial_n"]&C_{n-1}\arrow[d,hook]\arrow[r,"\partial{n-1}"]&\cdots\\
		B_{\bullet-1}\arrow[d]&\cdots\arrow[r]&B_n\arrow[r,"0"]&B_{n-1}\arrow[r,"0"]&\cdots\\
		0&&&
	\end{tikzcd}\]
	Los homomorfismos frontera son $0$ cuando vemos este complejo de cadenas como subcomplejo de cadenas de $\C_\bullet$. Y algo parecido para $B_{\bullet-1}$. Ahora, la sucesión exacta corta de complejos de cadenas se escinde y al dualizar obtenemos
	\[\begin{tikzcd}
		0\arrow[r]&B_{\bullet-1}^*\arrow[r]&C^*_\bullet\arrow[r]&Z_\bullet^*\arrow[r]&0
	\end{tikzcd}\]
	Ahora sí, aplicamos el teorema fundamental del álgebra homológica para obtener
	\[\begin{tikzcd}
		\cdots\arrow[r]&B_{n-1}^*\arrow[r]&H^n(C_\bullet,G)\arrow[r]&Z_n^*\arrow[r]&B_n^*\arrow[r]&H^{n+1}(C_\bullet,G)\arrow[r]&\cdots
	\end{tikzcd}\]
	\begin{af}
		$Z_n^*\xrightarrow{i_n^*} B_n^*$ dado por $g\mapsto g|_{B_n}$ es el dual de $B_n\hookrightarrow Z_n$.
	\end{af}
	\begin{obs}
		El dual de una inclusión es una restricción.
	\end{obs}
	En la prueba simplemente mostramos que el mapeo $g\mapsto \bar{g}\partial_{n+1}:B_n\to G$ es una restricción.
	\[\begin{tikzcd}
		0\arrow[d]&&0&0&\\
		Z_\bullet\arrow[d,hook]&\cdots\arrow[r]&Z^*_n\arrow[r,"0",red]\arrow[u]&Z^*_{n-1}\arrow[u,red]\arrow[r,"0"]&\cdots\\
		C_\bullet\arrow[d]&\cdots\arrow[r]&C^*_n\arrow[u,red]\arrow[r,"\partial^*_n"]&C_{n-1}\arrow[u]\arrow[r,"\partial^*_{n-1}"]&\cdots\\
		B_{\bullet-1}\arrow[d]&\cdots\arrow[r]&B^*_n\arrow[r,"0"]\arrow[u]&B^*_{n-1}\arrow[u]\arrow[r,"0"]&\cdots\\
		0&&&
	\end{tikzcd}\]
	Perseguimos ese diagrama para construir el diagrama conmutativo
	\[\begin{tikzcd}
		&Z_n\arrow[r,"g"]\arrow[d,hook]&G\\
		B_n\arrow[ur,hook]\arrow[r,hook]&C_n\arrow[ur,"\bar{g}"]\\
		&C_{n+1}\arrow[ul]\arrow[u]
	\end{tikzcd}\]
	\textit{Parece que la $\bar{g}$ cambió de ser una restricción a una extensión...}
	
	Luego
	\[\begin{tikzcd}
		0\arrow[r]&\coker i_{n-1}^*\arrow[r]&H^n(C_\bullet;G)\arrow[r,"h"]&\ker i_{n}^*\arrow[r]&0
	\end{tikzcd}\]
	\begin{af}
		$\ker i^*_n=\Hom(H_n(C_\bullet),G)$
	\end{af}
	\begin{af}
		$\ker h=\coker i_{n-1}^*$.
	\end{af}
	Entonces,
	\[\begin{tikzcd}
		Z_{n-1}^*\arrow[r,"i_{n-1}^*"]&B_{n-1}^*\arrow[r]&\coker i_{n-1}^*\arrow[r]&0
	\end{tikzcd}\]
	Y ahora nada más tomamos
	\[\begin{tikzcd}
		&&&0\\
		0\arrow[r]&B_{n-1}\arrow[r,hook]&Z_{n-1}\arrow[r]\arrow[ur]&H_{n-1}(C_\bullet)\arrow[r]&0
	\end{tikzcd}\]
	Y de aquí que
	\[\begin{tikzcd}
		0\arrow[dr]&&&0\\
		&Z_{n-1}^*\arrow[r,"i_{n-1}^*"]&B_{n-1}^*\arrow[r]\arrow[ur]&\coker i_{n-1}^*\arrow[r]&0
	\end{tikzcd}\]
	Y esa de cuatro no es exacta pero sí concluimos que $\coker i_{n-1}^*=\Ext^1(H_n(C_\bullet),G)$.
\end{proof}
\begin{teo}[de coeficientes universales]
	Sean $G$ un grupo abeliano y $C_\bullet$ un $\Z$-complejo de cadenas de grupos abelianos libres, entonces tenemos la sucesión exacta
	\[\begin{tikzcd}
		0\arrow[r]&\Ext^1(H_{n-1}(C_\bullet),G)\arrow[r]&H^n(C_\bullet,G)\arrow[r,"h"]&\Hom_\Z(H_n(C),G)\arrow[r]&0
	\end{tikzcd}\]
	que se escinde.
	
	Además, esta sucesión es \textbf{natural} en $C_\bullet$, es decir, si $	\alpha:C_\bullet\to C'_\bullet$, entonces el siguiente diagrma conmuta:
	\[\begin{tikzcd}
		0\arrow[r]&\Ext^1(H_{n-1}(C_\bullet),G)\arrow[r]&H^n(C_\bullet,G)\arrow[r,"h"]&\Hom_\Z(H_n(C),G)\arrow[r]&0\\
		0\arrow[r]&\Ext^1(H_{n-1}(C'_\bullet),G)\arrow[r]\arrow[u,"\alpha_*"]&H^n(C'_\bullet,G)\arrow[r,"h'"]\arrow[u,"\alpha^*"]&\Hom_\Z(H_n(C'),G)\arrow[r]\arrow[u,"\alpha_*"]&0
	\end{tikzcd}\]
	donde estamos abusando de notación con las flechas inducidas por $\alpha$, que son distintas, pero se obtienen usando que $H_n$ y $\Ext$ son funtores.
\end{teo}
\begin{obs}
	El nombre \textit{coeficientes universales} tiene que ver con que la homología con coeficientes en $\Z$ determina la cohomología con cualesquiera coeficientes.
\end{obs}
\begin{proof}
	Ahora demostremos la naturalidad. Comenzamos con el siguiente diagrama:
	\[\begin{tikzcd}
		0\arrow[r]&Z_\bullet\arrow[r]\arrow[d,"\alpha"]&C_\bullet\arrow[r,"\partial"]\arrow[d,"\alpha"]&B_{\bullet,-1}\arrow[r]\arrow[d,"\alpha"]&0\\
		0\arrow[r]&Z'_\bullet\arrow[r]&C_\bullet'\arrow[r]&B'_{\bullet-1}\arrow[r]&0
	\end{tikzcd}\]
	Luego dualizamos:
	\[\begin{tikzcd}
		0\arrow[r]&Z^*_\bullet\arrow[r]&C^*_\bullet\arrow[r]&B^*_{\bullet-1}\arrow[r]&0\\
		0\arrow[r]&Z^{'*}_\bullet\arrow[r]\arrow[u,"\alpha^*"]&C^{'*}_\bullet\arrow[r]\arrow[u,"\alpha^*"]&B^{'*}_{\bullet,-1}\arrow[r]\arrow[u,"\alpha^*"]&0
	\end{tikzcd}\]
	Y por último
	\[\begin{tikzcd}
		0\arrow[r]&\coker i^*_{n-1}\arrow[r]&H^n(C_\bullet,G)\arrow[r]&\ker i^*_n\arrow[r]&0\\
		0\arrow[r]&\coker i^{'*}_{n-1}\arrow[r]\arrow[u,"\alpha^*"]&H^n(C'_\bullet,G)\arrow[r]\arrow[u,"\alpha^*"]&\ker i^{'*}_n\arrow[r]\arrow[u,"\alpha^*"]&0
	\end{tikzcd}\]
\end{proof}
\begin{coro}
	Supongamos que $H_n(C_\bullet)$ y $H_{n-1}(C_\bullet)$ son finitamente generados. Podemos expresarlos así:
	\[H_n(C_\bullet)=\Z^{r_n}\oplus T_n\qquad\qquad H_{n-1}(C_\bullet)=\Z^{r_{n-1}}\oplus T_{n-1}\]
	Con $T_n$ y $T_{n-1}$ finitos. Entonces,
	\[H^n(C_\bullet,\Z)\approx\Z^{r_n}\oplus T_{n-1}\]
\end{coro}
\begin{proof}
	\begin{align*}
		H^n(C_\bullet,\Z)&\approx\Hom(H_n(C_\bullet,\Z))\oplus\Ext_\Z^1(H_{n-1}(C_\bullet),\Z)\\
		&\approx\Hom(\Z^r\oplus T_n,\Z)\oplus\Ext(Z^{r_{n-1}}\oplus T_{n-1},\Z)\\
		&\approx\Hom(\Z^{r_n},\Z\oplus\Hom(T_n,\Z)\oplus\Ext(\Z^{r_{n-1}},\Z)\oplus\Ext(T_{n-1},\Z))\\
		&\approx \Z^{r_n}\oplus T_{n-1}
	\end{align*}
\end{proof}
\begin{coro}
	Si $\alpha:C_\bullet\to C_\bullet'$ induce isomorfismos en homología, entonces induce isomorfismos en cohomología.
\end{coro}
\begin{proof}
	En la prueba de la naturalidad del teorema de coeficientes universales, los $\alpha_*$ de los extremos son isomorfismos. Por el lema de los cinco, $\alpha_*$ también lo es.
\end{proof}

\chapter{Cohomología}
\section{Cohomología de espacios}
Sea $X$ un espacio topológico y $G$ un grupo abeliano. Podemos definir el complejo de cadenas singulares de $X$,
\[\begin{tikzcd}
	C_\bullet(X)&\cdots\arrow[r]&C_{n+1}\arrow[r,"\partial{n-1}"]&C_n\arrow[r,"\partial_n"]&C_{n-1}\arrow[r]&\cdots
\end{tikzcd}\]
donde 
	\[C_n(X):=\left\{\sum_{i=1}^mr_i\sigma_i|m\in\Z,r_i\in\R,\sigma_i\text{ es un simplejo singular}\right\}\approx\bigoplus_{n\text{ sim. sing.}}\Z\]
Y definimos
\begin{align*}
	H_n(X;\Z)&=H_n(C_\bullet(X))\\
	C^\bullet(X;G)&=\Hom(C_\bullet(X),G)\\
	H^n(X;G)&=H_n(C^\bullet(X,G))
\end{align*}
de donde
\[\begin{tikzcd}
	\cdots&\Hom(C_{n+1},G)\arrow[l]&\Hom(C_n,G)\arrow[l,swap,"\partial_{n+1}^*"]&\Hom(C_{n-1},G)\arrow[l,swap,"\partial_n^*"]&\cdots\arrow[l]
\end{tikzcd}\]
Y las fronteras están definidas mediante
\[\begin{tikzcd}
	C_n\arrow[r,"\varphi"]&G\\
	C_{n+1}\arrow[u,"\partial_{n+1}"]\arrow[ur,dashed]
\end{tikzcd}\]
Y definimos los cociclos y las cofronteras.

Y el \textbf{teorema de coeficientes universales en cohomología} nos dice que:
\[\begin{tikzcd}
	0\arrow[r]&\Ext^1(H_{n-1}(X),G)\arrow[r]&H^n(X,G)\arrow[r,"h"]&\Hom_\Z(H_n(X),G)\arrow[r]&0
\end{tikzcd}\]
También podemos definir los \textbf{grupos de cohomología reducidos} comenzando con el complejo de cadenas
\[\begin{tikzcd}
	\cdots\arrow[r]&C_{n+1}\arrow[r,"\partial{n-1}"]&C_n\arrow[r,"\partial_n"]&C_{n-1}\arrow[r]&\cdots\arrow[r]&C_0\arrow[r,"\varepsilon"]&\Z\arrow[r]&0
\end{tikzcd}\]
dualizando, y calculando homología. Como con la homología reducida, $\tilde{H}^n(X;G)=H^n(X;G)$, y $H^0(X)\approx\Hom(H_0(X),G)\approx\bigoplus_{\pi_0(X)}G$. Para verlo, usamos el teorema de coeficientes universales. Y si $X$ es arco-conexo, por la propiedad universal del abelianizado y usando el homomorfismo de Hurewics, $H^n(X)\approx\Hom(\pi_1(X),G)$.

También hay un \textbf{teorema de coeficientes universales para la cohomología reducida}.
	
Ahora tratemos de definir la cohomología de la pareja. Todo comienza usando
\[\begin{tikzcd}
	0\arrow[r]&C_\bullet(A)\arrow[r]&C_\bullet(X)\arrow[r]&C_\bullet(A,X)\arrow[r]&0
\end{tikzcd}\]
Que se escinde ya que la base de $C_\bullet(A)$ es una sub-base de $C_\bullet(X)$. Aplicamos $\Hom(-,G)$ para obtener
\[\begin{tikzcd}
	0\arrow[r]&C_\bullet^*(X,A)\arrow[r]&C_\bullet^*(X)\arrow[r]&C_\bullet^*(A)\arrow[r]&0
\end{tikzcd}\]
Y aquí calculamos los grupos de homología
\[H^n(X,A):=H_n(C_\bullet^*(X,A))\]
Aplicamos el teorema fundamental del álgebra para obtener
\[\begin{tikzcd}
	\cdots\arrow[r]&H^n(X,A)\arrow[r]&H^n(X)\arrow[r]&H^n(A)\arrow[r]&H^{n+1}(X,A)\arrow[r]&\cdots
\end{tikzcd}\]
Y también tenemos \textbf{naturalidad}, es decir, si tenemos $f:(X,A)\to(Y,B)$ es decir, una función continua con $f(A)\subset B$, entonces el siguiente diagrama conmuta:
\[\begin{tikzcd}
	0\arrow[r]&C_\bullet(A)\arrow[r]\arrow[d,"f_\#"]&C_\bullet(X)\arrow[r]\arrow[d,"f_\#"]&C_\bullet(A,X)\arrow[r]\arrow[d,"f_\#"]&0\\
	0\arrow[r]&C_\bullet(B)\arrow[r]&C_\bullet(Y)\arrow[r]&C_\bullet(Y,B)\arrow[r]&0
\end{tikzcd}\]
Luego dualizamos,
\[\begin{tikzcd}
	0\arrow[r]&C^*_\bullet(A)\arrow[r]&C^*_\bullet(X)\arrow[r]&C^*_\bullet(A,X)\arrow[r]&0\\
	0\arrow[r]&C^*_\bullet(B)\arrow[r]\arrow[u,"f^*_\#"]&C^*_\bullet(Y)\arrow[r]\arrow[u,"f^*_\#"]&C^*_\bullet(Y,B)\arrow[r]\arrow[u,"f^*_\#"]&0
\end{tikzcd}\]
Y también aquí todo conmuta:
\[\begin{tikzcd}
	\cdots\arrow[r]&H^n(X,A)\arrow[r]&H^n(X)\arrow[r]&H^n(A)\arrow[r]&H^{n+1}(X,A)\arrow[r]&\cdots\\
	\cdots\arrow[r]&H^n(Y,B)\arrow[r]\arrow[u]&H^n(Y)\arrow[r]\arrow[u]&H^n(B)\arrow[r]\arrow[u]&H^{n+1}(Y,B)\arrow[r]\arrow[u]&\cdots
\end{tikzcd}\]
Aprovechamos para decir que una función de parejas como ésta tiene una \textbf{función inducida}
\[f^*:H^n(Y,B)\to H^n(X,A)\]
y de hecho $(fg)^*=g^*f^*$, $id^*_{(X,A)}=id_{H^n(X,A)}$. O sea que de verdad tenemos un funtor.

También tenemos \textbf{invarianza homotópica}, es decir, si $f,g:(X,A)\to(Y,B)$ son homotópicas, entonces $f^*=g^*$. Recordemos un poco la demostración: si dos funciones de parejas son homotópicas, entonces las funciones inducidas en los complejos de cadenas son homotópicas con la definición algebraica, así que inducen iguales en la homología. Dualizando ese diagrama con flechas diagonales de colores, obtenemos que las funciones en cocadenas son homotópicas, así que serán iguales en cohomología.

También tenemos un \textbf{teorema de coeficientes universales para parejas}, que se enuncia así:
\[\begin{tikzcd}
	0\arrow[r]&\Ext_\Z^1(H_n(X,A),G)\arrow[r]&H^n(X,A;G)\arrow[r]&\Hom(H_n(X,A),G)\arrow[r]&0
\end{tikzcd}\]
que se puede aplicar simplemente porque el complejo de cadenas $C_\bullet(X,A)$ es de grupos abelianos libres.

Y además el siguiente diagrama conmuta:
\[\begin{tikzcd}
	0\arrow[r]&\Ext^1(H_{n-1}(X,A),G)\arrow[r]&H^n(X,A;G)\arrow[r,"h"]&\Hom_\Z(H_n(X,A),G)\arrow[r]&0\\
	0\arrow[r]&\Ext^1(H_{n-1}(Y,B),G)\arrow[r]\arrow[u,"\alpha_*"]&H^n(Y,B;G)\arrow[r,"h'"]\arrow[u,"\alpha^*"]&\Hom_\Z(H_n(Y,B),G)\arrow[r]\arrow[u,"\alpha_*"]&0
\end{tikzcd}\]

Y también hay \textbf{escisión}: si $Z\subseteq A\subseteq X$ tal que $\bar{Z}\subseteq A$, $i: (X-Z,A-Z)\hookrightarrow (X,A)$, entonces $i^*:H^n(X-Z,A-Z)\xrightarrow{\approx}H^n(X,A)$, cuya demostración se puede dar con escisión homológica y coeficientes universales vía lema de los cinco.

Ahora tomemos la cuña de espacios tales que el punto de pegado tiene una vecindad contraible, entonces:
\begin{align*}
	\tilde{H}_n\left(\bigvee_\alpha X_\alpha\right)\approx\bigoplus_\alpha H_n(X_\alpha)\qquad\tilde{H}^n\left(\bigvee_\alpha X_\alpha\right)\approx\prod_\alpha\tilde{H}^n(X_\alpha)\\
	H_n\left(\bigsqcup_\alpha X_\alpha\right)\approx\bigoplus H_n(X_\alpha)\qquad H^n\left(\bigsqcup_\alpha X_\alpha\right)\approx\prod_\alpha H^n(X_\alpha)
\end{align*}
Para complejos CW,
\[H_n(X^{n},X^{n-1})\approx H_n(X^{n}/X^{n-1})\approx H_n\left(\bigvee_{n\text{-células}}S^n\right)\approx\bigoplus_{n\text{-células}}\Z\]
Recordemos que la homología celular se construye de acuerdo al siguiente diagrama:
\[\begin{tikzcd}[column sep=tiny]
	& H_{n-1}(X^{n-1}) \arrow[rd, "j_{n-1}"] \\
	H_n(X^n, X^{n-1}) \arrow[ru, "\partial_n"] \arrow[rr, "d_n"] && H_{n-1}(X^{n-1}, X^{n-2}) \arrow[rd,swap, "\partial_{n-1}"] \arrow[rr, "d_{n-1}"] && H_{n-2}(X^{n-2}, X^{n-3}) \\
	& && H_{n-2}(X^{n-2}) \arrow[ru,swap, "j_{n-2}"]
\end{tikzcd}\]
Recordemos que los mapeos $j_{n-1}$ y $\partial_{n-1}$ son parte de la sucesión exacta larga de la pareja, de modo que su composición es cero, así que las $d_i$ son cero porque factorizan ese cero.

Dualicemos esto:
\[\begin{tikzcd}[column sep=tiny]
	& H^{n-1}(X^{n-1})\arrow[ld,swap, "\partial^*_n"] \\
	H^n(X^n, X^{n-1})   && H^{n-1}(X^{n-1}, X^{n-2})\arrow[ll,swap, "D^n"]\arrow[lu,swap, "j^*_{n-1}"]  && H^{n-2}(X^{n-2}, X^{n-3})  \arrow[ld, "j^*_{n-2}"]\arrow[ll,swap, "D^{n-1}"] \\
	& && H^{n-2}(X^{n-2})\arrow[lu, "\partial^*_{n-1}"]
\end{tikzcd}\]
Obtenemos funciones $D_n$ en vez de las $d_n$, y pues nada, tenemos \textbf{chomología celular}, y otra vez,
\begin{prop}
	 Si $X$ es un complejo CW, $H_n(C_{CW}^\bullet(X)):=H_{CW}^n(X)\approx H^\bullet(X)$, y además $C_{CW}^\bullet(X;G)$ es el dual de $C_\bullet^{CW}(X;G)$.
\end{prop}
\begin{proof}
		 Recordemos que
	 \begin{align*}
	 	\Hom(\bigoplus_{n\text{-células}}\Z,G)&=\prod_{n\text{-células}}\Hom(\Z,G)\\
	 	&=\prod_{n\text{-células}}G
	 \end{align*}
	 De forma que los grupos de cohomología son
	 \[H^m(X^n,X^{n-1})=
	 \begin{aligned}
	 	\begin{cases}
	 	\prod_{n\text{-células}}G\qquad&\text{si }n=m\\
	 	0\qquad&\text{si }n\neq m
	 \end{cases}
	 \end{aligned}\]
	 Ya que los grupos de homología son libres.
	 
	 Ahora observemos que $H^k(X^n)\approx H^k(X^{n-1})$ para $k\neq n,n-1$. Fijando $k$ y bajando la $n$, esto implica que $H^k(X^n)\approx H^k(X^0)\approx0$. De aquí que podemos poner algunos ceros en el diagrama de arriba:
	\[\begin{tikzcd}[column sep=tiny]
		&&&0\arrow[ld,blue]\\
		0&&\textcolor{blue}{H^{n-1}(X^n)}\arrow[ld,blue]\\
		&H^{n-1}(X^{n-1})\arrow[lu]\arrow[ld,swap, "\partial^*_n"] \\
		H^n(X^n, X^{n-1})   && H^{n-1}(X^{n-1}, X^{n-2})\arrow[ll,swap, "D^n"]\arrow[lu,swap, "j^*_{n-1}"]  && H^{n-2}(X^{n-2}, X^{n-3})  \arrow[ld, "j^*_{n-2}"]\arrow[ll,swap, "D^{n-1}"] \\
		& && H^{n-2}(X^{n-2})\arrow[lu, "\partial^*_{n-1}"]\arrow[ld]\\
		&&0
	\end{tikzcd}\]
	Ahora notemos que para $k\leq n+1$, $H^k(X,X^{n+1})=0$ ya que
	\[H^k(X,X^{n+1})\approx\Hom(H_k(X,X^{n+1}))\oplus\Ext(H_{k-1}(X,X^{n+1}))\]
	Luego, tenemos una sucesión exacta corta:
	\[\begin{tikzcd}
		0\approx H^n(X,X^{n+1})\arrow[r]&H^n(X;G)\arrow[r,"\approx"]&H^n(X^{n+1};G)\arrow[r]&H^{n+1}(X,X^{n+1})\approx0
	\end{tikzcd}\]
	Así que ese $H^{n-1}(X^n)$ que agregamos en el diagrama de arriba es $\approx H^{n-1}(X)$.
	
	Por fin, las flechas en la diagonal arriba-izquierda del diagrama son parte de la sucesión exacta de la pareja, de la cual obtenemos que
	\begin{align*}
		H^{n-1}(X)&\approx\ker\partial_{n-1}\\
		&\approx\ker D^{n-1}/\img\partial_{n-2}\\
		&\approx\ker D^{n-1}/\img D^{n-2}\\
		&\approx\Hom^{n-1}(X)
	\end{align*}
	Con lo que demostramos la primera parte. (Hay que recorrer la $n$ en todos lados para que sea más claro, pero así está bien).
	
	Para lo segundo, tomemos el siguiente diagrama, que resulta de dualizar otro
	\[\begin{tikzcd}
		H^n(X^n,X^{n-1})\arrow[r,"j_n"]\arrow[d,"h"]&H^n(X^n)\arrow[r,"\partial_n"]\arrow[d,"h"]&H^{n+1}(X^{n+1},X^n)\arrow[d,"h"]\\
		\Hom(H_n(X^n,X^{n1}))\arrow[r]&\Hom(H_n(X^n))\arrow[r]&\Hom(H_{n+1}(X^{n+1},X^n))
	\end{tikzcd}\]
	En primer lugar, por los $\Ext$, las $h$ de los extremos son isomorfismos. Además, por las sucesiones exactas largas de las parejas, los cuadrados conmutan, de forma que las composiciones de las flechas horizontales son isomorfas, que es justo lo que queríamos demostrar.
\end{proof}
Por último,
\begin{teo}
	Mayer-Vietoris en cohomología.
\end{teo}
\section{Producto copa}
Esto es que hace distinta la cohomología de la homología. En algún punto de esta sección necesitaremos que los coeficientes estén en un anillo conmutativo con unidad $R$. Construiremos el anillo
\[\left(\bigoplus_{n\geq0}H^n(X),+,\smile\right)\]
\begin{defn}
	Recordemos que $C^n(X)=\Hom(C_n(X),R)$, y tomemos dos cocadenas $\varphi\in C^k(X,R)$ y $\psi\in C^\ell(X,R)$. El \textbf{producto copa} es
	\[\varphi\smile\psi\in C^{k+\ell}(X;R)\]
	que debe ser una función definida en las $k+\ell$ cadenas:
	\[(\varphi\smile\psi)(\sigma:\Delta^{k+\ell}\to X):=\varphi(\sigma|[v_0,\ldots,v_k])\cdot_R\psi(\sigma|[v_{k},\ldots,v_{k+\ell}])\]
	Que es como aplicar la $\phi$ en los primeros $k$ vértices del simplejo y luego aplicar $\psi$ en los últimos $\ell$. Cada uno me da un elemento en $R$ y los multiplico.
\end{defn}
Ya con esto tenemos
\[\left(\bigoplus_{n\geq0}C^n(X;R),+,\smile\right)\]
Ya que el producto copa es asociativo y distributivo casi por definición. Además, la cocadena $1\in C^0(X;R)$ es una unidad.
\begin{obs}
	Se trata de un \textbf{anillo graduado}, que es una estructura de la forma $(\bigoplus_{i\in\N\cup\{0\}}A_i,+,\cdot)$ tal que $A_i$ es un grupo abeliano y para cualesquiera $a\in A_i$ y $b\in A_j$, $a\cdot b\in A_{i+j}$.
\end{obs}
Para pasar a homología, lo único que debemos verificar es que este producto se restringe bien a ciclos y pasa bien al cociente:
\begin{prop}
	$\partial(\varphi\smile\psi)=\partial\varphi\smile\psi+(-1)^k\varphi\smile\partial\psi$
\end{prop}
\begin{proof}
	Simplemente calcular cada componente. Parece que la alternancia viene del operador frontera.
\end{proof}
De lo cual deducimos que
\begin{itemize}
	\item ciclo$\smile$ciclo=ciclo.
	\item frontera$\smile$ciclo=frontera, ya que el término $\partial\varphi\smile\psi$ se anula.
	\item ciclo$\smile$frontera=frontera.
\end{itemize}
Y deducimos el producto copa en la homología no depende de los representantes para obtener el \textbf{anillo de cohomología}
\[\left(\bigoplus_{n\geq0}H^n(X;R),+,\smile\right)\]
Denotaremos $H^*(X,R):=\bigoplus_{n\geq0}H^n(X;R)$.

Ahora debemos demostrar que el mapeo que asocia a un espacio $X$ este anillo es funtorial. Esto ya se tiene cuando vemos el anillo de cohomología como un grupo. Basta demostrar que
\begin{prop}
	$f^*(\varphi\smile\psi)=f^*\varphi\smile f^*\psi$.
\end{prop}
\begin{prop}[Anticonmutatividad]
	Si $\alpha\in H^k$ y $\beta\in H^\ell$,
	\[\alpha\smile\beta=(-1)^{k+\ell}\beta\smile\alpha\]
\end{prop}
Observemos que $\alpha\smile\alpha:=\alpha^2=-\alpha^2\implies2\alpha^2=0$ no necesariamente implica que $\alpha=0$, y que $H^*(X,\Z/2)$ sí es conmutativo.
\begin{ejem}[Anillo de cohomología de $\R P^n$]
	Calculamos la homología de $\R P^n$ con coeficientes en $\Z$:
	\[\begin{tikzcd}
		\Z\arrow[r]&\cdots\arrow[r]&\Z\arrow[r,"2"]&\Z\arrow[r,"0"]&\Z\arrow[r,"0"]&0
	\end{tikzcd}\]
	y dualizamos con $\Hom(-,\Z/2)$:
	\[\begin{tikzcd}
		\Z/2&\Z/2\arrow[l,swap,"0"]&\Z/2\arrow[l,swap,"0"]&0\arrow[l,swap,"0"]
	\end{tikzcd}\]
	Así que la cohomología es $\Z/2$ si $0\leq i\leq n$ y $0$ en $i>n$. El producto de la clase de grado uno en el primer grupo de cohomología consigo misma me va dando los elementos no cero en cada nivel (esto es lo único que no está demostrado en este ejemplo). Es decir, $\alpha_1^i=\alpha_i$ para cualquier $2\leq i\leq n$ y $\alpha^{n+1}_1=0$
	
	 Entonces
	\[H^*(\R P^n,\Z/2)\approx\bigoplus_{i=0}^n\Z/2\approx\Z/2[X]/\langle X^{n+1}\rangle\]
	usando la relación que pusimos arriba.
	
	\begin{prop}\leavevmode
		\begin{enumerate}
			\item $H^*(\bigsqcup X_i,R)\approx\prod H^*(X_i,R)$.
			\item $\widetilde{H^*}(\bigvee X_i,R)=\prod \widetilde{H^*}(X,R)$ ya que, como esperaríamos, existe un anillo de cohomología reducida, aunque no necesariamente tiene unidad.
		\end{enumerate}
	\end{prop}
	\begin{proof}
		Rutina.
	\end{proof}
	\begin{ejem}
		Ahora comparemos $S\vee S^2$ y $\R P^2$, cuyos anillos de cohomología son los dos $\Z/2\oplus\Z/2$. Pero el elemento no trivial de grado 1, $\beta_1$ elevado al cuadrado es 0 porque estamos en el producto directo de los anillos, así que estamos en el producto del primer anillo, es decir
		\[(\beta_1,0)^2=(\beta_1^2,0)=(0,0)\]
		pero, como dijimos antes, el análogo en el proyectivo al cuadrado es el elemento no trivial en grado 2.
	\end{ejem}
\end{ejem}
\section{Teorema de Künneth}
Recordemos que si $X$ y $Y$ son complejos CW, su producto $X\times Y$ también tiene estructura de complejo CW: su $n$-esqueleto es $C^{CW}_n(X\times Y)=\Z\{e^i\times e^j:i+j=n\}$.

\subsection{Producto tensorial}
Sea $R$ un anillo conmutativo con 1, y $M,N$ $R$-módulos. Una función $f:M\times N\to A$ es \textbf{$R$-bilineal} si es lineal en cada entrada. El \textbf{producto tensorial} de $M$ y $N$ sobre $R$ es un $R$-módulo $M\otimes_RN$ con una función bilineal tal que se cumple la siguiente propiedad universal
\[\begin{tikzcd}
	(m,n)\arrow[r,maps to]&m\otimes n\\
	M\times N\arrow[r]\arrow[rd,swap,"\text{bilineal}"]&M\otimes_RN\arrow[d,dashed,"R\text{-lineal}"]\\
	&A
\end{tikzcd}\]
\begin{align*}
	M\otimes_RN=R[m\otimes n|m\in M,n\in N]\Big/\Big\langle &(m_1+m_2)\otimes n-m_1\otimes n-m_2\otimes n,\\
	&m\otimes(n_1+n_2)-m\otimes n_1-m\otimes n_2,\\
	&(rm)\otimes n-r(m\otimes n),\\
	&m\otimes(rn)-r(m\otimes n)\Big\rangle
\end{align*}
\begin{prop}[Propiedades del producto tensorial]\leavevmode
	\begin{itemize}
		\item $R\otimes_RM\approx M$.
		\item $M\otimes_RN\approx N\otimes_RM$ (ésta podría tener detalles si $R$ no es conmutativo).
		\item $\left(\bigoplus_{i\in I}M_i\right)\otimes_R N\approx\bigoplus_{i\in I}(M_i\otimes_RN)$.
		\item Si $M_1\xrightarrow{f}M_2$ es $R$-lineal, la función $M_1\otimes_RN\xrightarrow{f\otimes_R\Id_N}M_2\otimes_RN$que manda $m_1\otimes n\mapsto f(m_1)\otimes n$ es $R$-lineal.
		\item Para $N$ fijo,
		\[\begin{tikzcd}
			&\Ab\\
			-\otimes N:\RMod\arrow[r]\arrow[ur]&\RMod\arrow[u,hook]
		\end{tikzcd}\]
		es covariante exacto derecho, es decir,
		\begin{tikzcd}
			0\arrow[r]&M''\arrow[r]&M\arrow[r]&M'\arrow[r]&0
		\end{tikzcd}
		induce
		\begin{tikzcd}
			M''\otimes N\arrow[r]&M\otimes N\arrow[r]&M'\otimes N\arrow[r]&0
		\end{tikzcd}. Y también tenemos $\Tor^i_R(-,N)$.
		
		\item Si $L_1$ y $L_2$ son $R$-módulos libres, entonces,
			\[L_1\otimes L_2=\left(\bigoplus_\alpha R\right)\otimes\left(\bigoplus_\beta R\right)=\bigoplus_{\alpha\times\beta}R\]
		que nos recuerda justamente a la construcción del producto de dos complejos CW.
		\item Si $A=\bigoplus_{i\in\N}A_i$ y $B=\bigoplus_{j\in\N}B_j$ son $R$-módulos graduados, su producto
		\[A\otimes_RB=\bigoplus_{i+j=n}A_i\otimes B_j\]
		también tiene estructura de módulo graduado.
		\[\begin{matrix}
			\vdots&&\\
			A_0\otimes B_n&&\\
			\vdots&&\\
			A_0\otimes B_2\\
			A_0\otimes B_1&A_1\otimes B_1&\\
			A_0\otimes B_0&A_1\otimes B_0&A_2\otimes B_0&\cdots&A_n\otimes B_0&\cdots
		\end{matrix}\]
		\item Si $X$ y $Y$ son complejos CW,
		\[C_*^{CW}(X\times Y)\approx C_*^{CW}(X)\otimes_RC_*^{CW}(Y)\]
		como $R$-módulos graduados, es decir, $C_n^{CW}(X\times Y)\approx\oplus_{i+j=n}C_i^{CW}(X)\otimes C_j^{CW}(Y)$ para toda $n$.
	\end{itemize}
\end{prop}
Y la pregunta natural para complejos CW es ¿qué pasa en la homología? Tendremos la siguiente fórmula que recuerda al teorema de coeficientes universales.
\[\begin{tikzcd}[column sep=small]
	0\arrow[r]&\bigoplus_{i+j=k}(H_i(X)\otimes H_j(Y))\arrow[r]&H_k(X\times Y;R)\arrow[r]&\bigoplus_{i+j=k-1}\Tor(H_i(X),H_j(Y))\arrow[r]&0
\end{tikzcd}\]
Aunque no es cierto que $H^*(X)\times H^*(Y)$ es igual que $H^*(X\times Y;R)$, podemos definir una función llamada \textbf{producto cruz} de la siguiente forma
\begin{align*}
	H^*(X)\times H^*(Y)&\to H^*(X\times Y;R)\\
	(a,b)&\mapsto p^*_1(a)\smile p_2^*(b)
\end{align*}
Donde $p_1:X\times Y\to X$ y $p_2:X\times Y\to Y$. Claramanete es una función bilineal.

Podemos extender al producto tensorial:
\[\begin{tikzcd}
	H^*(X)\times H^*(Y)\arrow[r,"\times"]\arrow[d]&H^*(X\times Y;R)\\
	H^*(X)\otimes H^*(Y)\arrow[dashed,ur,"\times",swap]
\end{tikzcd}\]
Definamos una estructura de anillo en $H^*(X)\otimes H^*(Y)$ mediante
\[(a\otimes b)(c\otimes d)=(-1)^{|b||c|}(ac)\otimes(bd)\]
con lo cual el mapeo de arriba $\times$ es un homomorfismo de anillos.
\begin{prop}[Künneth]
	El mapeo $\times$ es un isomorfismo de anillos si $H^*(Y,R)$ es un $R$-módulo libre.
\end{prop}
\begin{proof}
	Checar en Hatcher.
\end{proof}
Ahora calculamos usando cohomología relativa el isomorfismo
\[H^*(X,A)\otimes H^*(Y,B)\to H^*(X\times Y,(A\times Y)\cup (X\times B))\]

\begin{prop}
		\[H^*(\R P^n,\Z/2)\approx\bigoplus_{i=0}^n\Z/2\approx\Z/2[X]/\langle X^{n+1}\rangle\]
\end{prop}
\begin{proof}
	Basta ver que el producto de los elementos no triviales en cada grado es el producto no trivial en la suma de los grados.
	
	Recordemos que
	\begin{align*}
		\R P^n&=\{(x_0,\ldots,x_n)\in\R^{n+1}\}\Big/x\sim \lambda y,\lambda\neq0\\
		&=D^n\Big/x\sim -x, x\in S^{n-1}
	\end{align*}
	Denotemos $\R P^n$ por $P^n$. Queremos ver que el elmento no trivial en $0\neq\alpha\in H^i(P^n)$ copa el elemento no trivial en $0\neq\beta\in H^j(P^n)$ es igual al elemento no trivial en ${0\neq\alpha\smile\beta\in H^{i+j}(P^n)}$.
	
	Consideremos los encajes naturales para $i+j=n$
	\begin{align*}
		\begin{aligned}
			P^i&\hookrightarrow P^n\\
		[x_0,\ldots,x_i]&\mapsto[x_0,\ldots,x_i,0,\ldots,0]
		\end{aligned}\qquad
		\begin{aligned}
			P^j&\hookrightarrow P^n\\
			[y_0,\ldots,y_j]&\mapsto[0,\ldots,0,y_0,\ldots,y_k]
		\end{aligned}
	\end{align*}
	Pensando en el modelo del disco, $P^i$ y $P^j$ son intersecciones de hiperplanos en el disco. Luego $P^i\cap P^j=\{p\}$.
	
	Notemos además que $\partial P^n=P^{n-1}$, y que $P^n-P^{n-1}$ es una bola abierta que denotaremos por $\R^n$. Y también, $P^j-P^{n-1}\approx\R^j$ y $P^i-P^{n-1}\approx\R^i$.
	\[\begin{tikzcd}
		H^i(P^n)\times H^j(P^n)\arrow[r,"\smile"]&H^n(P^n)\\
		H^i(P^n,P^n-P^j)\times(P^n,P^n-P^i)\arrow[u,"1"]\arrow[r,"\smile"]\arrow[d,"3",swap]&H^n(P^n,P^n-\{p\})\arrow[u,"2",swap]\arrow[d,"4"]\\
		H^i(\R^n,\R^n-\R^j)\times H^j(\R^n,\R^n-\R^j)\arrow[r,"\smile"]&H^n(\R^n,\R^n-\{p\})
	\end{tikzcd}\]
	Incluyendo las parejas usando $(P^n,\varnothing)$. El resultado se sigue de mostrar que las flechas verticales son isomorfismos y que en el renglón de hasta abajo, la copa de generadores va a dar al generador.
	
	Es posible retraer la pareja $(\R^n,\R^n-\R^j)$ en $(\R^i,\R^i-\{0\})$ de manera que
	\[\begin{tikzcd}
		H^i(\R^n,\R^n-\R^j)\times H^j(\R^n,\R^n-\R^i)\arrow[r,"\smile"]\arrow[d]&H^n(\R^n,\R^n-\{0\})\arrow[d]\\
		H^i(\R^i,\R^i-\{0\})\times H^j(\R^j,\R^j-\{0\})\arrow[r,"\times"]&H^n(\R^n,\R^n-\{0\})
	\end{tikzcd}\]
	Observemos que la cohomología de $H^i(\R^i,\R^i-\{0\})$ es la misma que la de la esfera ya que podemos proyectar radialmente y la cohomología de $\R^n$ se anula en la sucesión exacta larga de la pareja.
	
	Luego, la cohomología del producto tensorial de $\tilde{H}^*(S^{j-1})\otimes \tilde{H}^*(S^{i-1})$ es justamente $\Z/2\otimes\Z/2\approx\Z/2$ ya que son los únicos que factores que sobreviven en todos los grados. Finalmente aplicamos Künneth en el renglón de abajo: el generador de $H^i(\R^i,\R^i-\{0\})$ copa el generador de $H^j(\R^j,\R^j-\{0\})$ me da el generador de $H^n(\R^n,\R^n-\{0\}$.
	
	Como dijimos, para terminar basta ver que las flechas verticales son isomorfismos. El 4 es una aplicación directa del teorema de escición. Para el 2, como podemos retraer $P^n-\{p\}$ en $P^{n-1}$, tenemos que $H^n(P^n,P^n-\{p\})\approx H^n(P^n,P^{n-1})$. Y éste último de hecho es la homología de una esfera, así que debe ser $\Z/2$, igual que $H^n(P^n)$.
	
	Para el 3, veamos que todos aquí son isomorfismos:
	\[\begin{tikzcd}
		H^i(P^n)\arrow[d,"1"]&H^i(P^n,P^{i-1})\arrow[l,"5",swap]\arrow[d,"2"]&H^i(P^n,P^n-P^j)\arrow[r,"7"]\arrow[d,"3"]\arrow[l,"6",swap]&H^i(\R^n,\R^n-\R^j)\arrow[d,"4"]\\
		H^i(P^i)&H^i(P^i,P^{i-1})\arrow[l,"8"]&H^i(P^i,P^i-\{p\})\arrow[l,"9"]\arrow[r,"10",swap]&H^i(\R^i,\R^i-\{0\})
	\end{tikzcd}\]
	La flecha 1 se sigue por inclusión y porque estamos con coeficientes en $\Z/2$. 4 es la retracción que ya habíamos visto. 5,6,8,9 por inclusiones. Creo que 10 por escición. 6 es porque $P^n-P^j$ se retrae or deformación a $P^{i-1}$, y las demás se siguen porque todo conmuta.
\end{proof}
\begin{obs}
	Podemos pensar que el producto copa mide cómo se intersectan dos subvariedades.
\end{obs}
\begin{obs}
	Si existe $f:X\to\R P^n$, entonces la función inducida $f^*:H^*(\R P^n;\Z/2)\to H^*(X;\Z/2)$ está determinada por su acción en el polinomio $x$.
\end{obs}
\begin{obs}\leavevmode
	\begin{itemize}
	\item $H^*(\R P^\infty)\approx\Z/2[X]$ con una demostración análoga.
	
	\item Para $\C P^n=e^0\cup e^2\cup e^4\cup\ldots\cup e^{2n}$, $H^*(\C P^n;R)\approx R[X]/\langle X^{n+1}\rangle$ para cualquier anillo, con el generador de orden 2. Y también el caso de inifinito sin dividir entre el ideal.
	
	\item Para los cuaternios, $H^*(\mathbb{H}P^n,R)\approx R[X]/\langle X^{n+1}\rangle$ con $|X|=4$.
	
	\item $H^*(\R P^\infty\times\R P^\infty,\Z/2)\approx\Z/2[X]\otimes\Z/2[Y]\approx\Z/2[X,Y]$.
	
	\item Iterando el inciso anterior, agregamos más y más variables. Y creemos que el caso infinito da el anillo de polinomios en infinitas variables.
	\end{itemize}
\end{obs}
Para calcular $H^*(\R P^n;\Z)$, tenemos por coeficientes universales la cohomología
\[\begin{tikzcd}
	\Z\arrow[r]&0\arrow[r]&\Z/2\arrow[r]&0\arrow[r]&\cdots\arrow[r]&\Z/2\qquad n\text{ par}\\
	\Z\arrow[r]&0\arrow[r]&\Z/2\arrow[r]&0\arrow[r]&\cdots\arrow[r]&\Z\qquad n\text{ impar}
\end{tikzcd}\]
Observemos que para cualquier espacio topológico $X$ y $R\to S$ homomorfismo de anillos, siempre tenemos el homomorfismo $H^i(X;R)\to H^i(X;S)$ para toda $i$. Además, este homomorfismo respecta el producto copa, de manera que tendremos un morfismo
\[H^*(P^n,\Z)\to H^*(P^n,\Z/2)\approx\Z/2[X]\]
Es decir tenemos el diagrama
\[\begin{tikzcd}
	\Z\arrow[r]\arrow[d,hook]&0\arrow[r]\arrow[d,hook]&\Z/2\arrow[r]\arrow[d,hook]&0\arrow[r]\arrow[d,hook]&\cdots\arrow[r]&\Z/2\text{ o }\Z\arrow[d,hook]\\
	\Z\arrow[r]&\Z/2\arrow[r]&\Z/2\arrow[r]&\Z/2\arrow[r]&\cdots\arrow[r]&\Z/2
\end{tikzcd}\]
donde las flechas verticales resultaron ser inclusiones, al menos en todas salvo la última columna. En estos casos, multiplicamos el emento no trivial de $Z/2$ en el renglón de arriba consigo mismo, lo que nos manda dos grados arriba a algún elemento. Para ver que éste no es el trivial, usamos el producto copa que ya definimos en el caso de coeficientes en $\Z/2$ y la simple definición de morfismo de anillos.

Para el caso de dimensión impar sólo notemos que la potencia del elemento no trivial en dimensión 2 no puede caer en el grupo de grado impar, así que necesitamos otro generador para el anillo de cohomología. En resumen:
\[H^*(P^n,\Z)\approx\begin{cases}
	\begin{aligned}
		\Z[X]/\langle2X,X^{n+1}\rangle\qquad\qquad&|X|=2,\text{ en dimensión par}\\
		\Z[X,Y]/\langle 2X,X^n,Y^n,Y\rangle\qquad&|X|=2,|Y|=n,\text{ en dimensión impar}
	\end{aligned}
\end{cases}\]
Una aplicación de esto es que para $m$ par,
\[\tilde{H}^*(\R P^m\vee S^{m+1},\Z)\approx\tilde{H}(\R P^m)\times\tilde{H}(S^{m+1})\]
Para verlo, primero notemos que, como grupos abelianos, $\tilde{H}^*(\R P^m\vee S^{m+1},\Z)\approx_{\text{ab}}\tilde{H}(\R P^m+1)$ ya que la cohomología de $S^{m+1}$ no aporta nada hasta la dimensión $m+1$, donde le agrega un a copia de $\Z$. Cuando consideramos todo el anillo, se vuelve el producto. Y además, este anillo resulta isomorfo al anillo de cohomología del espacio proyectivo de dimensión impar.

El mismo ejemplo con coeficientes en $\Z/2$ tiene el mismo isomorfismo de grupos abelianos, pero no de anillos.
\end{document}